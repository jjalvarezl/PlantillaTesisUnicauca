%input macros (i.e. write your own macros file called MacroFile1.tex)
%\include{Macros/MacroFile1}

 \documentclass[oneside,11pt]{Classes/CUEDthesisPSnPDF}
 

\ifpdf
    \pdfinfo { /Title  (Titulo de mi tesis)
               /Creator (TeX)
               /Producer (pdfTeX)
               /Author (Jhon Jairo Alvarez Londoño jjalvarezl@unicauca.edu.co)
               /CreationDate (D:20140511000000)  %format D:YYYYMMDDhhmmss
               /ModDate (D:20131015135448)
               /Subject (De que se trata la tesis)
               /Keywords (Mi palabra clave 1; Mi palabra clave 2; ...)}
    \pdfcatalog { /PageMode (/UseOutlines)
                  /OpenAction (fitbh)  }
\fi

\title{Titulo de mi tesis}

%\ifpdf
%  \author{\href{mailto:jhon.jairo.alvarez.londono@hotmail.com}{Jhon Jairo Alvarez Londoño}}
%  \collegeordept{\href{http://portal.unicauca.edu.co/versionP/acerca-de-unicauca/facultades/facultad-de-ingenieria-electronica-y-telecomunicaciones}{Departmento de Sistemas de la Información}}
%  \university{\href{http://www.unicauca.edu.co}{Universidad del Cauca}}
% insert below the file name that contains the crest in-place of 'UnivShield'
%  \crest{\includegraphics[width=30mm]{EscudoUnicauca}}
%\else
  \degree{Grado al que aspiro}
  \author{\href{mailto:jjalvarezl@unicauca.edu.co}{Jhon Jairo Alvarez Londoño}}
  \thesistutor {Director: Tutor de la tesis}
  \college{Facultad}
  \department{Departamento}
  \university{Universidad del Cauca}
  \researchgroup{Nombre del grupo de investigación extendido}
  \researchgroupname{Nombre acortado del grupo de investigación}
  \researchgroupinfo{Especialidad de la tesis}
  \date{Popayán, \today}
% insert below the file name that contains the crest in-place of 'UnivShield'
  \crest{\includegraphics[viewport = 0 0 292 336, width=30mm]{EscudoUnicauca}}
%  \crest{\includegraphics[bb = 0 0 292 336, width=30mm]{EscudoUnicauca}}
%\fi
%
% insert below the file name that contains the crest in-place of 'UnivShield'
% \crest{\IncludeGraphicsW{UnivShield}{40mm}{14 14 73 81}}
%
%\renewcommand{\submittedtext}{change the default text here if needed}


% turn of those nasty overfull and underfull hboxes
% \hbadness=10000
% \hfuzz=50pt

% Put all the style files you want in the directory StyleFiles and usepackage like this:
%\usepackage{StyleFiles/watermark}

% Comment out the next line to get single spacing
%\onehalfspacing
\begin{document}
\urlstyle{rm}
%\bstctlcite{IEEEexample:BSTcontrol} % Ver el control "IEEEexample" en el archivo
% .bib (inhabilita el "----" de los autores repetidos).
%\language{english}

% A page with the abstract on including title and author etc may be
% required to be handed in separately. If this is not so, then comment
% the below 3 lines (between '\begin{abstractseparte}' and 
% 'end{abstractseparate}'), normally like a declaration ... needs some more
% work, mind as environment abstracts creates a new page!
% \begin{abstractseparate}
%   \input{Abstract/abstract}
% \end{abstractseparate}




% Using the watermark package which is in StyleFiles/
% and to remove DRAFT COPY ONLY appearing on the top of all pages comment out below line
%\watermark{DRAFT COPY ONLY}
\maketitle

%set the number of sectioning levels that get number and appear in the contents
\setcounter{secnumdepth}{3}
\setcounter{tocdepth}{3}

\frontmatter % book mode only
\pagenumbering{roman}
\include{Dedication/dedication}
\include{Acknowledgement/acknowledgement}
\include{Abstract/abstract}

\tableofcontents
\listoffigures
\listoftables

%\printnomenclature  %% Print the nomenclature
%\addcontentsline{toc}{chapter}{Nomenclature}

\mainmatter % book mode only
\setlength{\parskip}{0.2cm} % This command changes the space between
%%% Thesis Introduction --------------------------------------------------
\chapter{Introducción}
\ifpdf
    \graphicspath{{Introduction/IntroductionFigs/PNG/}{Introduction/IntroductionFigs/PDF/}{Introduction/IntroductionFigs/}}
\else
    \graphicspath{{Introduction/IntroductionFigs/EPS/}{Introduction/IntroductionFigs/}}
\fi

Coloque la introducción de su tesis, las siguientes secciones es como debería estructurar las principales secciones de soporte de su tesis.

\section{Planteamiento del problema}

\section{Pregunta de investigación}

\section{Objetivos}

\subsection{Objetivo general}

\subsection{Objetivos específicos}

\section{Método de exploración y evaluación}

Se define un método de exploración y evaluación inicial con el fin de cumplir los objetivos y responder las preguntas de investigación se define los marcos metodológicos de investigación


%%% Local Variables: 
%%% mode: latex
%%% TeX-master: "../thesis"
%%% End: 

\include{Chapter1/chapter1}
\chapter{Resto de mi tesis}
\ifpdf
    \graphicspath{{Chapter2/Chapter2Figs/PNG/}{Chapter2/Chapter2Figs/PDF/}{Chapter2/Chapter2Figs/}}
\else
    \graphicspath{{Chapter2/Chapter2Figs/EPS/}{Chapter2/Chapter2Figs/}}
\fi

Espacio para colocar el resto de la tesis... puede duplicarse este capitulo cuantas veces sea necesario para cubrir toda la estructura de la investigación.

% ------------------------------------------------------------------------


%%% Local Variables: 
%%% mode: latex
%%% TeX-master: "../thesis"
%%% End: 

\def\baselinestretch{1}
\chapter{Conclusiones, limitaciones, trabajo futuro y aportes adicionales}
\ifpdf
    \graphicspath{{Conclusions/ConclusionsFigs/PNG/}{Conclusions/ConclusionsFigs/PDF/}{Conclusions/ConclusionsFigs/}}
\else
    \graphicspath{{Conclusions/ConclusionsFigs/EPS/}{Conclusions/ConclusionsFigs/}}
\fi

\def\baselinestretch{1.66}

Aquí puede colocar una introducción a las conclusiones.

\section{Conclusiones}

Mi sección de conclusiones

\section{Limitaciones}

Mi sección de limitaciones

\section{Trabajos futuros}

Mi sección de trabajos futuros

\section{Aportes adicionales}

Mi sección de aportes adicionales en caso de que la tesis aplique a esta categoría.

%%% ----------------------------------------------------------------------

% ------------------------------------------------------------------------

%%% Local Variables: 
%%% mode: latex
%%% TeX-master: "../thesis"
%%% End: 


\bibliographystyle{IEEEtran}
%\bibliographystyle{plainnat}
%\bibliographystyle{Classes/CUEDbiblio}
%\bibliographystyle{Classes/jmb}
%\bibliographystyle{Classes/jmb} % bibliography style
\renewcommand{\bibname}{Bibliografía} % changes default name Bibliography to Bibliografía
%\bibliography{References/references} % References file
\bibliography{IEEEabrv,References/references}

%\backmatter % book mode only
\appendix
\makeatletter
\renewcommand{\@chapapp}{Anexo} % Si se quita esta linea o la anterior el capitulo no se llamará anexo sino apendice
\include{Appendix1/appendix1}
\end{document}
